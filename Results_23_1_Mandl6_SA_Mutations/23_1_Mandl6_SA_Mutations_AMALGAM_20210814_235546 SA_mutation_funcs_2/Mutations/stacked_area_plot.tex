\documentclass[crop=false]{standalone}
%https://tex.stackexchange.com/questions/288373/how-to-draw-stacked-area-chart
\usepackage{pgfplots}
\usepgfplotslibrary{dateplot}
\pgfplotsset{compat=1.12}
\usetikzlibrary{fillbetween}
\usetikzlibrary{external}
\tikzexternalize[prefix=savedfigures/]

\begin{document}
\pgfkeys{/pgf/number format/.cd,1000 sep={\,}}
\tikzsetnextfilename{Stacked_area_mut}
\begin{tikzpicture}[]
    \begin{axis}[
    legend style={at={(1.1,1)},anchor=north west}, %(<x>,<y>)
    %date coordinates in=x,
    table/col sep=comma,
    %xticklabel style={rotate=90, anchor=near xticklabel},
    ymin=0,
    ymax=1,
    %restrict y to domain=0:1,
    %max space between ticks=20,
    stack plots=y,%
    area style,each nth point=100, filter discard warning=false, unbounded coords=discard,
    xmin=0,
    xmax=4362,
    xlabel = Total iterations,
    ylabel = Selection probability
    ]

\addplot [draw=blue, fill=blue!30!white] table [mark=none, x=Total_iterations,y=Trim_one_terminal_cb
] {Smoothed_Mutation_Ratios_for_Tikz.csv}
    \closedcycle;
    \addlegendentry{Cost-based trim};

\addplot [draw=red, fill=red!30!white] table [mark=none, x=Total_iterations,y=Grow_one_terminal_cb
] {Smoothed_Mutation_Ratios_for_Tikz.csv}
    \closedcycle;
    \addlegendentry{Cost-based grow};

    \end{axis}
    \end{tikzpicture}
\end{document}